\documentclass[UTF8]{ctexart}
\usepackage{graphicx}
\usepackage{amsmath}
\usepackage{listings}
\usepackage{geometry}
\geometry{a4paper,scale=0.80}
\usepackage[dvipsnames]{xcolor}
\CTEXsetup[format={\large\bfseries}]{section}
%由小到大分别是: 
%\tiny \scriptsize \footnotesize \small \normalsize 
%\large \Large \LARGE \huge \Huge
\usepackage{xeCJK} % CJK语言环境,使用XeLaTex进行编译
\usepackage{authblk} % 对应中文部分的作者机构特殊语法
\author{asandstar@github}
\title{HW3}
\begin{document}
\maketitle
\section{简述估计后验概率的两种策略。}
 %表格生成网站https://www.tablesgenerator.com/latex_tables 
 (1)判别式模型($discriminative$\, $models$):
给定数据$\textbf{\emph{x}}$,
通过直接拟合$P(c|\textbf{\emph{x}})$
预测$\textbf{\emph{x}}$

换言之,利用正负例和分类标签,关注判别模型的边缘分布,不考虑x与y间的联合分布。

目标函数直接得到分类准确率。

例如:决策树、BP神经网络、支持向量机

(2)生成式模型($generative$\, $models$):
先对联合概率分布$P(\textbf{\emph{x}},c)$建模,
再据此得到$P(c|\textbf{\emph{x}})$
\\
\\
对生成式模型,考虑$P(c|\textbf{\emph{x}})=$
$\displaystyle{\frac{P(\textbf{\emph{x}},c)}{P(\textbf{\emph{x}})}}$
\\
\\
由贝叶斯定理,$P(c|\textbf{\emph{x}})=$
$\displaystyle{\frac{P(c)P(\textbf{\emph{x}}|c)}{P(\textbf{\emph{x}})}}$

$P(c|\textbf{\emph{x}})$:后验概率

$P(c)$:类“先验”概率

$P(\textbf{\emph{x}}|c)$:样本$\textbf{\emph{x}}$相对类标记$c$的类条件概率,即“似然”

$P(\textbf{\emph{x}})$:用于归一化的“证据”因子。(对给定样本$x$,该因子与类标记无关)
\\
估计$P(c|\textbf{\emph{x}})$的问题→
基于训练数据估计先验$P(c)$和似然$P(\textbf{\emph{x}}|c)$
\\
类先验$P(c)$:样本空间中各类样本占比。

由大数定律,训练集包含足够独立同分布样本,P(c)由各类样本出现频率估计
\\
似然$P(\textbf{\emph{x}}|c)$:涉及关于$\textbf{\emph{x}}$所有属性的联合概率,难以按出现频率直接估计

常用策略:假定其由某种确定的概率分布,再基于训练样本估计概率分布的参数(如极大似然估计MLE)

$D_c$:训练中第c类样本组合的数据集合

设样本独立,则$\theta _c$对$D_c$的似然:
\begin{center}
  $P(D_c|\theta _c)=\prod \limits_{x \in D_c}P(\textbf{\emph{x}}|\theta _c)$
\end{center}

对数似然:

\begin{center}
  $LL(\theta _c)=log\,P(D_c|\theta _c)=\sum \limits_{x \in D_c}P(\textbf{\emph{x}}|\theta _c)$
\end{center}

参数$\theta_c$的极大似然估计$\hat{\theta_c}$:

\begin{center}
  $\hat{\theta_c}=arg \mathop{max}\limits_{\theta_c}LL(\theta _c)$
\end{center}

\pagestyle{plain}
\section{根据下表数据,使用朴素贝叶斯分类器分别判别
  (红色、圆形、大苹果)和(青色、非规则形状、小
  苹果)是否为好果(注:可使用拉普拉斯修正)。\\}
\begin{table}[!h]
  \centering
  \setlength{\tabcolsep}{10mm}
  \begin{tabular}{|c|c|c|c|}
    \hline
    \textbf{大小} & \textbf{颜色} & \textbf{形状} & \textbf{好果} \\ \hline
    小            & 青            & 非规则        & 否            \\ \hline
    大            & 红            & 非规则        & 是            \\ \hline
    大            & 红            & 圆            & 是            \\ \hline
    大            & 青            & 圆            & 否            \\ \hline
    大            & 青            & 非规则        & 否            \\ \hline
    小            & 红            & 圆            & 是            \\ \hline
    大            & 青            & 非规则        & 否            \\ \hline
    小            & 红            & 非规则        & 否            \\ \hline
    小            & 青            & 圆            & 否            \\ \hline
    大            & 红            & 圆            & 是            \\ \hline
  \end{tabular}
\end{table}
1.估计类先验概率$P(c)$\\

$P$(好果=是)=$\displaystyle{\frac{2}{5}=0.4}$,
$P$(好果=否)=$\displaystyle{\frac{3}{5}=0.6}$\\


2.估计每个属性的条件概率$P(x_i|c)$
%红 圆 大 青 非规则 小

$P_{\mbox{\zihao{6}青|是}}=P$(颜色=青|好果=是)=$\frac{0}{4}=0$

显然不合理,进行“拉普拉斯修正”\\


3.重新估计类先验概率$P(c)$\\


$P$(好果=是)=$\displaystyle{\frac{4+1}{10+2}=\frac{5}{12}}$,$P$(好果=否)=$\displaystyle{\frac{6+1}{10+2}=\frac{7}{12}}$\\

4.估计每个属性的条件概率$P(x_i|c)$

$P_{\mbox{\zihao{6}青|是}}=P$(颜色=青|好果=是)=$\frac{0+1}{4+2}=\frac{1}{6}$

$P_{\mbox{\zihao{6}青|否}}=P$(颜色=青|好果=否)=$\frac{5+1}{6+2}=\frac{6}{8}$

$P_{\mbox{\zihao{6}红|是}}=P$(颜色=红|好果=是)=$\frac{4+1}{4+2}=\frac{5}{6}$

$P_{\mbox{\zihao{6}红|否}}=P$(颜色=红|好果=否)=$\frac{1+1}{6+2}=\frac{2}{8}$

$P_{\mbox{\zihao{6}圆|是}}=P$(形状=圆|好果=是)=$\frac{3+1}{4+2}=\frac{4}{6}$

$P_{\mbox{\zihao{6}圆|否}}=P$(形状=圆|好果=否)=$\frac{2+1}{6+2}=\frac{3}{8}$

$P_{\mbox{\zihao{6}非规则|是}}=P$(形状=非规则|好果=是)=$\frac{1+1}{4+2}=\frac{2}{6}$

$P_{\mbox{\zihao{6}非规则|否}}=P$(形状=非规则|好果=否)=$\frac{4+1}{6+2}=\frac{5}{8}$

$P_{\mbox{\zihao{6}大|是}}=P$(大小=大|好果=是)=$\frac{3+1}{4+2}=\frac{4}{6}$

$P_{\mbox{\zihao{6}大|否}}=P$(大小=大|好果=否)=$\frac{3+1}{6+2}=\frac{4}{8}$

$P_{\mbox{\zihao{6}小|是}}=P$(大小=小|好果=是)=$\frac{1+1}{4+2}=\frac{2}{6}$

$P_{\mbox{\zihao{6}小|否}}=P$(大小=小|好果=否)=$\frac{3+1}{6+2}=\frac{4}{8}$\\

5.判断是否为好果

对(红色、圆形、大苹果)

$P$(好果=是)$
  \,×\,P_{\mbox{\zihao{6}红|是}}
  \,×\,P_{\mbox{\zihao{6}圆|是}}
  \,×\,P_{\mbox{\zihao{6}大|是}}
  =\frac{5}{6}\,×\,\frac{4}{6}\,×\,\frac{4}{6}
  \approx 0.370$

$P$(好果=否)$
  \,×\,P_{\mbox{\zihao{6}红|否}}
  \,×\,P_{\mbox{\zihao{6}圆|否}}
  \,×\,P_{\mbox{\zihao{6}大|否}}
  =\frac{2}{8}\,×\,\frac{3}{8}\,×\,\frac{4}{8}
  \approx 0.047$

因为0.563>0.028,所以(红色、圆形、大苹果)是好果


对(青色、非规则形状、小苹果)

$P$(好果=是)$\,×\,P_{\mbox{\zihao{6}青|是}}
  \,×\,P_{\mbox{\zihao{6}非规则|是}}
  \,×\,P_{\mbox{\zihao{6}小|是}}
  =\frac{1}{6}\,×\,\frac{2}{6}\,×\,\frac{2}{6}
  \approx 0.019$

$P$(好果=否)$\,×\,P_{\mbox{\zihao{6}青|否}}
  \,×\,P_{\mbox{\zihao{6}非规则|否}}
  \,×\,P_{\mbox{\zihao{6}小|否}}
  =\frac{6}{8} \,×\,\frac{5}{8}\,×\,\frac{4}{8}
  \approx 0.234$

因为0.019<0.234,所以(青色、非规则形状、小苹果)不是好果

\pagestyle{plain}
\section{使用半朴素贝叶斯分类器中的SPODE方法, 对于下表
  数据,假定$x_2$为超父,试预测$x_1=1,x_2=1,x_3=0$时$y=1$的
  概率。\\}
\begin{table}[!h]
  \centering
  \setlength{\tabcolsep}{10mm}
  \begin{tabular}{|c|c|c|c|}
    \hline
    $x_1$ & $x_2$ & $x_3$ & $y$ \\ \hline
    1     & 1     & 1     & 1   \\ \hline
    1     & 0     & 0     & 1   \\ \hline
    1     & 1     & 1     & 1   \\ \hline
    1     & 0     & 0     & 0   \\ \hline
    1     & 1     & 1     & 0   \\ \hline
    0     & 0     & 0     & 0   \\ \hline
    0     & 1     & 1     & 0   \\ \hline
    0     & 1     & 0     & 1   \\ \hline
    0     & 1     & 1     & 0   \\ \hline
    0     & 0     & 0     & 0   \\ \hline
  \end{tabular}
\end{table}
半朴素贝叶斯的基本想法:考虑一部分属性间的相互依赖信息,
从而既不需进行完全联合概率计算,又不至于彻底忽略了比较强的属性依赖关系
\\
独依赖估计($One-Dependent \, Estimator$,简称$O D E $):半朴素贝叶
斯分类器最常用的一种策略
\\
独依赖:假设每个属性在类别之外最多仅依赖于一个其他属性
\begin{center}
  $P(c|\textbf{\emph{x}})\propto P(c) \prod \limits_{i=1}^d P(x_i|c,pa_i)$

  $pa_i$是$x_i$的父属性,即属性$x_i$所依赖的属性
\end{center}

问题的关键→如何确定每个属性的父属性

$SPODE (Super-Parent \, ODE)$方法:假设所有属性都依赖于同一个属性 “超父”(super-parent), 然后通过交叉验证等模型选择方法来确定超父属性

此时$y=arg \mathop{max}P(c) P(x_j|c) \prod \limits_{i=1,i\neq j}^d P(x_i|c,x_j)$,其中$x_j$是超父

$P(y=1)=\frac{4}{10}$

假定超父是 $x_1$,那么,对于$ y=1 $的可能性有:

$P(x_1=1|y=1)=\frac{3}{4}$

$P(x_2=1|y=1,x_1=1)=\frac{2}{3}$

$P(x_3=0|y=1,x_1=1)=\frac{1}{3}$

$P_1=P(y=1) P(x_1|y=1) \prod \limits_{i=1,i\neq j}^d
  P(x_i|y=1,x_1)=\frac{4}{10}  \,×\, \frac{3}{4} \,×\,
  \frac{2}{3} \,×\,\frac{1}{3}=\frac{1}{15}$

假定超父是 $x_2$,那么,对于$ y=1 $的可能性有:

$P(x_2=1|y=1)=\frac{3}{4}$

$P(x_1=1|y=1,x_2=1)=\frac{2}{3}$

$P(x_3=0|y=1,x_2=1)=\frac{1}{3}$

$P_2=P(y=1) P(x_2|y=1) \prod \limits_{i=1,i\neq j}^d
  P(x_i|y=1,x_2)= \frac{4}{10} \,×\,\frac{3}{4} \,×\,
  \frac{2}{3}\,×\,\frac{1}{3}=\frac{1}{15}$

假定超父是 $x_3$,那么,对于$ y=1 $的可能性有:

$P(x_3=0|y=1)=\frac{2}{4}$

$P(x_1=1|y=1,x_3=1)=\frac{1}{2}$

$P(x_2=1|y=1,x_3=1)=\frac{1}{2}$

$P_3=P(y=1) P(x_3|y=1) \prod \limits_{i=1,i\neq j}^d
  P(x_i|y=1,x_3)=  \frac{4}{10}\,×\, \frac{2}{4} \,×\,
  \frac{1}{2} \,×\,\frac{1}{2}=\frac{1}{20}$

因此$x_1$和$x_2$均可作为超父,且$x_1=1,x_2=1,x_3=0$时$y=1$的概率为$\frac{1}{15}$

\pagestyle{plain}
\section{给出如下贝叶斯网络, 在一个人呼吸困难
  (Dyspnoea)的情况下, 其抽烟(Smoking)的概率是
  多少。}
\includegraphics[width = .9\textwidth]{bgraph.jpg}

由图可以得出$P(S,L,B,X,D)=P(S)P(L|S)P(B|S)P(X|S,L)P(D|L,B)$

%2.$S$已知的情况下,$L$和$B$和$X$是独立的$(tail-to-tail)$

%3.$L$已知的情况下,$X$和$D$是独立的$(tail-to-tail)$

%4.$B$已知的情况下,$S$和$D$是独立的$(head-to-tail)$

%5.$D$未知的情况下,$L$和$B$是独立的$(head-to-head)$

%6.$X$未知的情况下,$L$和$B$是独立的$(head-to-head)$\\


%$P(S|D=1)=\frac{P(S,D=1)}{P(D=1)}\propto P(S,D=1)$

%$\displaystyle{P(S|D=1)=\frac{P(S,D=1)}{P(D=1)}\propto P(S,D=1)}$
%$=P(S)\sum \limits_{D=1}\sum \limits_{B}P(B|S)
%  \sum \limits_{X}\sum \limits_{L} P(L|S) P(X|S,L)  P(D|L,B)$

在一个人呼吸困难 (Dyspnoea) 的情况下, 其抽烟 (Smoking) 的概率为
\begin{center}
  $\displaystyle{P(S=1|D=1)=\frac{P(S=1,D=1)}{P(D=1)}}$\\
\end{center}

$P(S=1,D=1)$
$=P(S=1)\sum \limits_{D=1}\sum \limits_{B}P(B|S=1)
  \sum \limits_{X}\sum \limits_{L} P(L|S=1) P(X|S=1,L)  P(D|L,B)$

$=0.5×\sum \limits_{D=1}\sum \limits_{B}P(B|S=1)
  \sum \limits_{X}\sum \limits_{L} P(L|S=1) P(X|S=1,L)  P(D|L,B)$

$=0.5×\sum \limits_{D=1}\sum \limits_{B}P(B|S=1)
  [ P(L=0|S=1) P(X=0|S=1,L=0)  P(D|L=1,B)$

      $+P(L=0|S=1) P(X=1|S=1,L=0)  P(D|L=1,B)$

      $+P(L=1|S=1) P(X=0|S=1,L=1)  P(D|L=1,B)$

      $+P(L=1|S=1) P(X=1|S=1,L=1)  P(D|L=1,B)]$

$=0.5×[ P(B=0|S=1)P(L=0|S=1) P(X=0|S=1,L=0)  P(D=1|L=1,B=0)$

$+P(B=1|S=1)P(L=0|S=1) P(X=0|S=1,L=0)  P(D=1|L=1,B=1)$

$+P(B=0|S=1)P(L=0|S=1) P(X=1|S=1,L=0)  P(D=1|L=1,B=0)$

$+P(B=1|S=1)P(L=0|S=1) P(X=1|S=1,L=0)  P(D=1|L=1,B=1)$

$+P(B=0|S=1)P(L=1|S=1) P(X=0|S=1,L=1)  P(D=1|L=1,B=0)$

$+P(B=1|S=1)P(L=1|S=1) P(X=0|S=1,L=1)  P(D=1|L=1,B=1)$

$+P(B=0|S=1)P(L=1|S=1) P(X=1|S=1,L=1)  P(D=1|L=1,B=0)$

$+P(B=1|S=1)P(L=1|S=1) P(X=1|S=1,L=1)  P(D=1|L=1,B=1)]$

$=0.5×[0.3\,×\,0.2\,×\, 0.1\,×\,0.8$
$+0.7\,×\,0.2\,×\,0.1\,×\,0.9$
$+0.3\,×\,0.2\,×\,0.9 \,×\,0.8$
$+0.7\,×\,0.2\,×\,0.9 \,×\,0.9$

$+0.3\,×\,0.8\,×\,0.2 \,×\,0.8$
$+0.7\,×\,0.8\,×\,0.2 \,×\,0.9$
$+0.3\,×\,0.8\,×\,0.8\,×\, 0.8$
$+0.7\,×\,0.8\,×\,0.8\,×\,0.9]$

$=0.435$\\


$P(B=1)=P(B=1|S)\,×\,P(S)$
$=P(B=1|S=0)\,×\,P(S=0)+P(B=1|S=1)\,×\,P(S=1)$

$=0.5\,×\,0.5+0.7\,×\,0.5=0.6$

$P(B=0)=P(B=0|S)\,×\,P(S)$
$=P(B=0|S=0)\,×\,P(S=0)+P(B=0|S=1)\,×\,P(S=1)$

$=0.5\,×\,0.5+0.3\,×\,0.5=0.4$

$P(L=1)=P(L=1|S)\,×\,P(S)$
$=P(L=1|S=0)\,×\,P(S=0)+P(L=1|S=1)\,×\,P(S=1)$

$=0.55\,×\,0.5+0.8\,×\,0.5=0.675$

$P(L=0)=P(L=0|S)\,×\,P(S)$
$=P(L=0|S=0)\,×\,P(S=0)+P(L=0|S=1)\,×\,P(S=1)$

$=0.45\,×\,0.5+0.2\,×\,0.5=0.325$\\


$P(D=1)=P(D=1|L,B)\,×\,P(L,B)=P(D=1|L,B)\,×\,P(L)\,×\,P(B)$

$=P(D=1|L=0,B=0)\,×\,P(L=0)\,×\,P(B=0)$
$+P(D=1|L=0,B=1)\,×\,P(L=0)\,×\,P(B=1)$

$+P(D=1|L=1,B=0)\,×\,P(L=1)\,×\,P(B=0)$
$+P(D=1|L=1,B=1)\,×\,P(L=1)\,×\,P(B=1)$

$=0.1\,×\,0.325\,×\,0.4$
$+0.7\,×\,0.325\,×\,0.6$

$+0.8\,×\,0.675\,×\,0.4$
$+0.9\,×\,0.675\,×\,0.6$

$=0.73$\\


$\displaystyle{P(S=1|D=1)=\frac{P(S=1,D=1)}{P(D=1)}=\frac{0.435}{0.73}\approx 0.596}$

综上,一个人呼吸困难的情况下,其抽烟的概率是0.596
\end{document}