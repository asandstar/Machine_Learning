\documentclass[UTF8]{ctexart}
\usepackage{graphicx}
\usepackage{amsmath}
\usepackage{listings}
\usepackage{geometry}
\geometry{a4paper,scale=0.76}
\usepackage[dvipsnames]{xcolor}
% 在导言区进行样式设置
\lstset{
    language=Python, % 设置语言
 basicstyle=\ttfamily, % 设置字体族
 breaklines=true, % 自动换行
 keywordstyle=\bfseries\color{NavyBlue}, % 设置关键字为粗体,颜色为 NavyBlue
 morekeywords={}, % 设置更多的关键字,用逗号分隔
 emph={self}, % 指定强调词,如果有多个,用逗号隔开
    emphstyle=\bfseries\color{Rhodamine}, % 强调词样式设置
    commentstyle=\itshape\color{black!50!white}, % 设置注释样式,斜体,浅灰色
    stringstyle=\bfseries\color{PineGreen!90!black}, % 设置字符串样式
    columns=flexible,
    numbers=left, % 显示行号在左边
    numbersep=2em, % 设置行号的具体位置
    numberstyle=\footnotesize, % 缩小行号
    frame=single, % 边框
    framesep=1em % 设置代码与边框的距离
}
\CTEXsetup[format={\large\bfseries}]{section}
%由小到大分别是: 
%\tiny \scriptsize \footnotesize \small \normalsize 
%\large \Large \LARGE \huge \Huge
\usepackage{xeCJK} % CJK语言环境,使用XeLaTex进行编译
\usepackage{authblk} % 对应中文部分的作者机构特殊语法
\author{asandstar@github}
\title{HW2}
\begin{document}
\maketitle
\section{现有下表所示训练数据,给定感知机模型
$z = sgn(w_{1}x + w_{2} y + b)$\\
参数取值为$w_{1}=1,w_{2}=1,b=-1$\\
步长$\eta =0.5$,试学习感知机权重,给出过程}
\begin{table}[!h]
    \centering
    \setlength{\tabcolsep}{10mm}
    \begin{tabular}{|l|l|l|}
        \hline
        x     & y     & z  \\ \hline
        -0.89 & -1.62 & -1 \\ \hline
        -0.30 & 0.96  & +1 \\ \hline
        1.49  & 0.25  & +1 \\ \hline
        0.06  & -0.68 & -1 \\ \hline
        -1.13 & 0.14  & -1 \\ \hline
        0.78  & 0.96  & +1 \\ \hline
    \end{tabular}
\end{table}
%表格生成网站https://www.tablesgenerator.com/latex_tables 
设$\textbf{\emph{w}}=(w_{1},w_{2},b),\mathbf{\textbf {\emph{X}}}^\top=(x,y,1)$
,模型$z = sgn(\textbf{\emph{w}}·\textbf {\emph{X}})$

其中已知$w_{1}=1,w_{2}=1,b=-1$,所以$z = sgn(x +  y -1)$

选择一个数据点$(x_{i},y_{i},z_{i})$,
如果分类错误则更新参数
$\textbf{\emph{w}}←\textbf{\emph{w}}+\eta \textbf{\emph{X}}_i z_{i}$,
%$b←b+\eta z_{i}$

(1)对$\mathbf{\textbf {\emph{X}}_1}^\top=(-0.89,-1.62,-1),
    \textbf{\emph{w}}=(1,1,-1)$,

代入得$z=sgn(-0.89+(-1.62)-1)=-1$,分类正确

(2)对$\mathbf{\textbf {\emph{X}}_2}^\top=(-0.30,0.96,+1),
    \textbf{\emph{w}}=(1,1,-1)$,

代入得$z=sgn(-0.30+0.96-1)=-1 \neq +1$,分类错误

更新参数$\textbf{\emph{w}}=(1,1,-1)+0.5× (-0.30,0.96,1)×(+1)
    =(0.85,1.48,-0.5)$

(3)对$\mathbf{\textbf {\emph{X}}_1}^\top=(-0.89,-1.62,-1),
    \textbf{\emph{w}}=(0.85,1.48,-0.5)$,

代入得$z=sgn(0.85×(-0.89)+1.48×(-1.62)-0.5)=-1$,分类正确

(4)对$\mathbf{\textbf {\emph{X}}_2}^\top=(-0.30,0.96,+1),
    \textbf{\emph{w}}=(0.85,1.48,-0.5)$,

代入得$z=sgn(0.85×(-0.30)+1.48×0.96-0.5)=+1$,分类正确

(5)对$\mathbf{\textbf {\emph{X}}_3}^\top=(1.49,0.25,+1),
    \textbf{\emph{w}}=(0.85,1.48,-0.5)$,

代入得$z=sgn(0.85×1.49+1.48×0.25-0.5)=+1$,分类正确

(6)对$\mathbf{\textbf {\emph{X}}_4}^\top=(0.06,-0.68,-1),
    \textbf{\emph{w}}=(0.85,1.48,-0.5)$,

代入得$z=sgn(0.85×0.06+1.48×(-0.68)-0.5)=-1$,分类正确

(7)对$\mathbf{\textbf {\emph{X}}_5}^\top=(-1.13,0.14,-1),
    \textbf{\emph{w}}=(0.85,1.48,-0.5)$,

代入得$z=sgn(0.85×(-1.13)+1.48×0.14-0.5)=-1$,分类正确

(8)对$\mathbf{\textbf {\emph{X}}_6}^\top=(0.78,0.96,+1),
    \textbf{\emph{w}}=(0.85,1.48,-0.5)$,

代入得$z=sgn(0.85×0.78+1.48×0.96-0.5)=+1$,分类正确

综上,感知机权重为$\textbf{\emph{w}}=(w_{1},w_{2},b)=(0.85,1.48,-0.5)$

\pagestyle{plain}
\section{现有以下数据\\
  $$
      x=
      \begin{bmatrix}
          1    & 2    & 0 \\
          3    & 1    & 0 \\
          2.5  & 2    & 0 \\
          -2   & -2   & 1 \\
          -2.7 & -1   & 1 \\
          -3   & -2.2 & 1
      \end{bmatrix}
  $$
  其中每行为一个样本,前2列为属性,第3列表示类
  别,试应用LDA算法计算投影后的数据点}
LDA的思想:给定训练样例集,设法将样例投影到一条直线上,
使得同类样例的投影点尽可能接近、异类样例的投影点尽可能远离;
在对新样本进行分类时,将其投影到同样的这条直线上,
再根据投影点的位置来确定新样本的类别。
\\
\\
设$\mu _i $为第i类的均值向量,例如$\mu _0=\mathbf{(x_0,y_0)}^\top $为类别0的样本中心点

$\mathbf{\textbf{\emph{w}}}^\top =(w_0,w_1)$

%$\mathbf{\textbf {\emph{w}}}  $为所投影到的直线,
%其中$\textbf {\emph{w}}=w_0/w_1$
%,$\textbf{\emph{x}} =\mathbf{(\emph{x',y'})}^\top$
%,$x'$为题干中$x$第一列,$y'$为题干中$x$第二列,故$y=w_0 x'+w_1 y'=0$

%投影后数据点求解方法:

%已知LDA直线为$y=\mathbf{\textbf {\emph{w}}^\top \textbf{\emph{x}}}$

%$x$是样本向量(列向量),
%如果投影到一条直线上$w$就是一个特征向量(列向量形式)
%或者多个特征向量构成的矩阵。

%$y$ 为投影后的样本点(列向量)


$\mathbf{\textbf {\emph{w}}}^\top \mu _0
    ,\mathbf{\textbf {\emph{w}}}^\top \mu _1$为两类样本的中心在直线上的投影

$\mathbf{\textbf {\emph{w}}}^\top \Sigma _0 \textbf{\emph{w}}
    ,\mathbf{\textbf {\emph{w}}}^\top \Sigma _1 \textbf{\emph{w}}$为投影所有的点到直线后,
两类样本的协方差

(1)使同类样例的投影点尽可能接近, 可以让同类样例投影点的协方差尽可能小,
即$\mathbf{\textbf {\emph{w}}}^\top \Sigma _0 \textbf{\emph{w}}
    +\mathbf{\textbf {\emph{w}}}^\top \Sigma _1 \textbf{\emph{w}} $
尽可能小

(2)使异类样例的投影点尽可能远离, 可以让类中心之间的距离尽可能大,
即$|| \mathbf{\textbf {\emph{w}}}^\top \mu _0-\mathbf{\textbf {\emph{w}}}^\top \mu _1||_2^2$

所以最大化的目标为$\displaystyle{J=\frac{|| \mathbf{\textbf {\emph{w}}}^\top \mu _0-\mathbf{\textbf {\emph{w}}}^\top \mu _1||_2^2}
        {\mathbf{\textbf {\emph{w}}}^\top \Sigma _0 \textbf{\emph{w}}
            +\mathbf{\textbf {\emph{w}}}^\top \Sigma _1 \textbf{\emph{w}} }
        =\frac{\mathbf{\textbf {\emph{w}}}^\top (\mu_0-\mu_1 )\mathbf{(\mu_0-\mu_1 )}^\top \textbf{\emph{w}}}
        {\mathbf{\textbf {\emph{w}}}^\top (\Sigma _0+\Sigma _1 )\textbf{\emph{w}} }
    }$
\\
\\
定义类内散度矩阵$\textbf{S}_w=\Sigma _0+\Sigma _1$

$
    =\sum\limits_{x\in X_{0}} (\textbf{\emph{x}} \mathbf {-\mu_0 )(\textbf{\emph{x}} -\mu_0)}^\top+
    \sum\limits_{x\in X_{1}}(\textbf{\emph{x}}  \mathbf {-\mu_1 )(\textbf{\emph{x}} -\mu_1)}^\top
$
\\
\\
定义类间散度矩阵$ \textbf{S}_b=\mathbf{(\mu_0-\mu_1 )(\mu_0-\mu_1 )}^\top$

则最大化的目标可化为$S_w$和$S_b$的广义瑞利商
$\displaystyle{J=\frac{\mathbf{\textbf {\emph{w}}}^\top
            \textbf{S}_w \textbf{\emph{w}}}
        {\mathbf{\textbf {\emph{w}}}^\top
            \textbf{S}_b \textbf{\emph{w}}}}$

注意到上式的分子和分母都是关于w的二次项,
因此上式的解与w的长度无关,只与其方向有关,所以令$\mathbf{\textbf {\emph{w}}}^\top
    \textbf{S}_w \textbf{\emph{w}}=1$

故最大化的目标等价于

\begin{center}
    $\mathop{min}\limits_{w} -\mathbf{\textbf {\emph{w}}}^\top
        \textbf{S}_b \textbf{\emph{w}}$

    $s.t. \mathbf{\textbf {\emph{w}}}^\top
        \textbf{S}_w \textbf{\emph{w}}=1$
\end{center}

利用拉格朗日乘子法得($\lambda$为拉格朗日乘子)

\begin{center}
    $L(w,\lambda )=-\mathbf{\emph{w}}^\top S_b w
        +\lambda (\mathbf{\emph{w}}^\top S_w w -1)$
\end{center}

对上式求导并令其为0得

\begin{center}
    $\displaystyle{\frac{\partial L}{\partial w}=-2S_bw+2\lambda S_w w=0}$
\end{center}

因为$\mathbf{\textbf {\emph{w}}}^\top \mu$为标量,
所以$\textbf{S}_b w=\mathbf{(\mu_0-\mu_1 )(\mu_0-\mu_1 )}^\top w$中的
$\mathbf{(\mu_0-\mu_1 )}^\top w$也为标量,
即$\textbf{S}_b w$的方向仅与$\mathbf{(\mu_0-\mu_1 )}$有关

不妨令$\textbf{S}_b w=\lambda \mathbf{(\mu_0-\mu_1 )}$
,代入上述求导所得等式得

\begin{center}
    $ w=S_w^{-1}\mathbf{(\mu_0-\mu_1 )}$
\end{center}

对$S_w$进行奇异值分解$S_w=\mathbf{U}\Sigma \mathbf{V}^\top$,
其中$\Sigma$为实对角矩阵,其对角线上的元素是$S_w$的奇异值

再求逆得$S_w^{-1}=\mathbf{V}\Sigma ^{-1}\mathbf{U}^\top$,并求得$w$


由题可知被投影的直线$l_1$斜率$\displaystyle{k=\frac{w_0}{w_1}}$

所以与该直线垂直的直线$l_2$斜率$\displaystyle{k'=-\frac{w_1}{w_0}}$

$l_2$代入数据点,再与$l_1$联立即可求得投影点
\\
\\
具体公式为 $\displaystyle{x'=\frac{x+y·k'}{(k')^2 +1}},
    y'=kx'$
\\
\\
利用python求解的代码如下
\\
% 正文区
% lstlisting环境
\begin{lstlisting}
    from statistics import mean
    import numpy as np
    import matplotlib.pyplot as plt
    '''1.导入数据'''
    xy = np.array([[1, 2], [3, 1], [2.5, 2], [-2, -2], [-2.7, -1], [-3, 1]])
    xy0 = np.array(xy[:3])  #3×2维
    xy_0 = np.array(xy[:3])
    xy1 = np.array(xy[3:])  #3×2维
    xy_1 = np.array(xy[3:])
    #数据按题干x摆放,第一列是x,第二列是y
    '''2.计算miu0,miu1'''
    x0 = mean(xy0[:, 0])
    y0 = mean(xy0[:, 1])
    x1 = mean(xy1[:, 0])
    y1 = mean(xy1[:, 1])
    miu0 = [x0, y0]
    miu1 = [x1, y1]
    '''3.计算类内散度矩阵Sw'''
    for i in range(3):
        xy0[i][0] = xy0[i][0] - x0
        xy0[i][1] = xy0[i][1] - y0
        xy1[i][0] = xy1[i][0] - x1
        xy1[i][1] = xy1[i][1] - y1
        #xy0从x中类别为0的矩阵变为x-μ的矩阵
    sig0 = sig1 = 0
    for i in range(3):
        sig0 = sig0 + np.outer(xy0[i], xy0[i].T)
        sig1 = sig1 + np.outer(xy1[i], xy1[i].T)
        #某类中每一个i的(x-miu)与(x-miu)^T先外积,再对i求和
    sw = sig0 + sig1  #2×2维
    '''4.Sb·w的方向'''
    miu01 = np.array([x0 - x1, y0 - y1])  #miu0-miu1
    '''5.得到w'''
    swni = np.linalg.inv(sw)  #求逆
    w = np.dot(swni, miu01)
    '''6.投影点公式'''
    k = w[0] / w[1]
    kt = -w[1] / w[0]
    xty = (xy[:, 0] + xy[:, 1] * k) / (k**2 + 1)
    yty = k * xty
    xyty = np.array([xty, yty])
    xyty = xyty.T
    print('投影后数据点:\n', xyty)
    '''7.画图'''
    #画出题干数据点、被投影的直线
    plt.scatter(xy_0[:, 0], xy_0[:, 1], c='b', label='+', marker='+')
    plt.scatter(xy_1[:, 0], xy_1[:, 1], c='r', label='-', marker='_')
    plt.plot([-3.5, 3.5], [-3.5 * k, 3.5 * k], label='y')
    #画出投影点与原数据点的连线
    for i in range(6):
        plt.plot([xy[i][0], xyty[i][0]], [xy[i][1], xyty[i][1]])
    plt.xlabel('X1', fontproperties='SimHei', fontsize=15, color='black')
    plt.ylabel('X2', fontproperties='SimHei', fontsize=15, color='black')
    plt.xlim(-4, 4)
    plt.ylim(-4, 4)
    ax = plt.gca()
    ax.set_aspect(1)
    plt.legend()
    plt.show()
\end{lstlisting}

求得结果为
$w=[3.30924855,1.84393064]^\top $

投影后数据点:

[[ 1.08730441  1.95135353]

    [ 1.13595088  2.03865794]

    [ 1.44268362  2.58914223]

    [-1.32422388 -2.37654599]

    [-1.06487504 -1.91110019]

    [-0.28556595 -0.51249688]]
\\
\\
python作图如下

\includegraphics[width = .9\textwidth]{Figure_1.png}
\section{现用梯度下降求解优化问题:\\
  \begin{center}
      $x^{*}=arg \mathop{min}\limits_{x} f(x)$\\
  \end{center}
  即迭代公式为\\
  \begin{center}
      $\displaystyle{x_{i+1}=x_{i}-\eta \frac{ \partial f }{ \partial x }}$\\
  \end{center}
  若f是二阶多项式,试证明迭代收敛的条件是$\displaystyle{\eta < 2\left(\frac{ \partial ^{2}f }{ \partial x ^{2}}\right)^{-1}}$
 }

设$f$的表达式为$f(x)=ax^2+bx+c$,
则$\displaystyle{\frac{ \partial f }{ \partial x }=2ax+b,
        \frac{ \partial ^{2}f }{ \partial x ^{2}}=2a}$

又$f(x)$有最小值,故$a>0$

因为$||\frac{ \partial f }{ \partial x }-\frac{ \partial f }{ \partial y }||=
    \big | \big|\, ||2ax+b || -||2ay+b ||\, \big |\big|$

\qquad \qquad \qquad \quad \,$\leq  ||2a(x-y)||$

\qquad \qquad \qquad \quad \,$\leq  2a||x-y||$

即存在$L=2a$,使$||f'(x)-f'(y)||\leq L||x-y||$

所以$f'$满足Lipschitz continuous条件,$f$符合Lipschitz continuous gradient

由Descent Lemma,$f(x) \leq f(y)+\frac{\partial f}{\partial x}(x-y)+\frac{L}{2}||x-y||^2$

在上式中把$x,y$换成$x_{i_1},x_i$得
$f(x_{i+1}) \leq f(x_i)+\frac{\partial f}{\partial x}(x_{i+1}-x_i)+\frac{L}{2}||x_{i+1}-x_i||^2$

又因为$\displaystyle{x_{i+1}-x_{i}=-\eta \frac{ \partial f }{ \partial x }}$,
%假设$\eta =2/L$,则$\displaystyle{x_{i+1}-x_{i}=-\frac{2}{L}\frac{ \partial f }{ \partial x }}$

代入上述Descent Lemma得到的不等式得

\begin{center}
    $f(x_{i+1}) \leq f(x_i)-\eta ||f'(x_i) ||^2+\frac{L}{2}||\eta f'(x_i)||^2$

    $=f(x_i)-(\eta-\frac{L}{2}\eta^2)||f'(x_i) ||^2$
\end{center}

要使迭代收敛,需要满足$\eta-\frac{L}{2}\eta^2>0$,解得$\displaystyle{\eta <\frac{2}{L}}$

而$\displaystyle{L=2a=\frac{ \partial ^{2}f }{ \partial x ^{2}}}$

所以迭代的收敛条件为$\displaystyle{\eta < 2\left(\frac{ \partial ^{2}f }{ \partial x ^{2}}\right)^{-1}}$

证毕
\section{简答:使用OvR和MvM将多分类任务分解为二分类任务求解时,
  试概述针对类别不平衡性处理的方法。}
类别不平衡学习的基本策略是rescaling(再缩放)。

以线性分类器为例,用$y=\mathbf{\textbf {\emph{w}}} ^\top \textbf {\emph{x}}+b$
对$\mathbf{\textbf {\emph{w}}} ^\top$
进行分类时,是用预测的y和某个阈值相比较,
实际上y是$\textbf {\emph{x}}$属于某类(正例或反例)的可能性

$\displaystyle{\frac{y}{1-y}}$指正例可能性和反例可能性的比值

阈值=0.5表明分类器认为真实正、反例可能性相同相同

分类器决策规则:$\displaystyle{\frac{y}{1-y}}>1$时,预测为正例
\\
训练集中正、反例数目不同时,令$m^+$为正例数,$m^-$为反例数,
故观测几率可以表示为$\displaystyle{\frac{m^+}{m^-}}$\\
\\
假设训练集是真实样本总体的无偏采样,则观测几率代表真实几率\\
一旦分类器的预测几率高于观测几率,判定为正例,
即$\displaystyle{\frac{y}{1-y}>\frac{m^+}{m^-}}$时,预测为正例\\
\\
但$\displaystyle{\frac{y}{1-y}}>1$是分类器使用的决策依据,所以对预测值进行调整,令

\begin{center}
    $\displaystyle{\frac{y'}{1-y'}=\frac{y}{1-y}×\frac{m^+}{m^-}}$
\end{center}

实际情况是,因为 “训练集是真实样本总体的无偏采样”的假设往往不成立,
不能用观测几率代表真实几率,现在主要有三类做法。

(1)“欠采样”(undersampling)

去除训练集里的反类样例中一些反例使得正、反例数目接近,再学习

(2)“过采样" (oversampling)

增加训练集里的正类样例中一些正例使得正、反例数目接近,再学习

(3) “阈值移动”(threshold-moving)

基于原始训练集进行学习,并在用训练好的分类器进行预测时
,把式子$\displaystyle{\frac{y'}{1-y'}=\frac{y}{1-y}×\frac{m^+}{m^-}}$
嵌入到其决策过程中

\section{请写出求解支撑矢量机优化问题的SMO算法流程,
其中在固定$\lambda _{i1}$和$\lambda _{i2}$以外的参数时,
推导下面原优化问题的闭式解$\lambda _{i1}^{*}$ 和$\lambda _{i2}^{*}$\\
\\
$\displaystyle{\mathop{max}\limits_{\lambda_{i}\geq 0}\left\{\sum_{i}\lambda _{i}-
    \frac{1}{2}
    (\sum_{i}\sum_{j}\lambda_{i}\lambda_{j}
    y_{i}y_{j}(\textbf{\emph{x}}_i·\textbf{\emph{x}}_j))\right\}}$\\
\\
$s.t.  \sum_{i} \lambda_{i} y_{i}=0$}
%SMO的基本思路是先固定$\lambda _i$之外的所有参数,然后求$\lambda _i$上的极值
因为存在约束$\sum_{i} \lambda_{i} y_{i}=0$,
如果固定$\lambda _{i1},\lambda _{i2}$之外其他变量,
则$\lambda _{i1},\lambda _{i2}$可由其他变量导出

\begin{center}
    $\lambda_{i1} y_1+\lambda_{i2} y_2
        =-\sum_{i\neq 1,2}\lambda _i y_i=\varsigma$

    $0 \leq \lambda_i \leq C,i=1,2$
\end{center}

故题干最优化子问题,省略不含$\lambda_{i1},\lambda_{i2}$的常数项后,可写成:

\begin{center}

    $\displaystyle{\mathop{min}\limits_{\lambda _{i1},\lambda _{i2}}
            W(\lambda_{i1},\lambda_{i2})=
            \frac{1}{2}(\textbf{\emph{x}}_1·\textbf{\emph{x}}_1)
            \lambda_{i1}^2
            +
            \frac{1}{2}(\textbf{\emph{x}}_2·\textbf{\emph{x}}_2)
            \lambda_{i2}^2
            +
            y_1 y_2 (\textbf{\emph{x}}_1·\textbf{\emph{x}}_2)
            \lambda_{i1}\lambda_{i2}}$
    \\
    $\displaystyle{-
            (\lambda_{i1}+\lambda_{i2})
            +
            y_1 \lambda_{i1} \sum_{i\neq 1,2}y_i\lambda_i
            (\textbf{\emph{x}}_i·\textbf{\emph{x}}_1)
            +
            y_2 \lambda_{i2} \sum_{i\neq 1,2}y_i\lambda_i
            (\textbf{\emph{x}}_i·\textbf{\emph{x}}_2)
        }$

\end{center}

$s.t. $\quad $\lambda_{i1} y_1+\lambda_{i2} y_2=\varsigma $

$0 \leq \lambda_i \leq C,i=1,2$
\\
求解上述两个变量的二次规划问题,先要分析约束条件,然后在该约束条件下求极小
\\
只有两个变量,约束可以用二维空间中的图形表示

\includegraphics[width = .9\textwidth]{hw2plot.png}
\\
因为不等式约束$0 \leq \lambda_i \leq C,i=1,2$
使$(\lambda_{i1},\lambda_{i2})$在$[0,C]×[0,C]$内
\\
且等式约束$\lambda_{i1} y_1+\lambda_{i2} y_2=\varsigma $
使$(\lambda_{i1},\lambda_{i2})$在平行于$[0,C]×[0,C]$的对角线的直线上
\\
所以要求是目标函数在一条平行于对角线的线段上的最优值
\\
\\
又因为$\lambda_{i1} y_1+\lambda_{i2} y_2=\varsigma $
,其中$\varsigma$为常数,所以上述两个变量的最优化问题实质上是单变量的最优化问题,
不妨设该问题是$\lambda_{i2}$的最优化问题
\\
\\
设问题的初始可行解为$\lambda_{i1},\lambda_{i2}$,
最优解$\lambda_{i1}^*,\lambda_{i2}^*$,
设在沿着约束方向未考虑不等式$0 \leq \lambda_i \leq C,i=1,2$
约束时$\lambda_{i2}$的最优解是$\lambda_{i2}^{*,unc}$
\\
由图可以看出,$L \leq \lambda_{i2}^* \leq H$,其中$L,H$是$\lambda_{i2}^*$所在的平行于对角线的线段端点的界

(1)$y_1 \neq y_2$,则
$L=max(0,\lambda_{i2}-\lambda_{i1})$,
$H=min(C,C+\lambda_{i2}-\lambda_{i1})$

(2)$y_1 \neq y_2$,则
$L=max(0,\lambda_{i2}+\lambda_{i1}-C)$,
$H=min(C,\lambda_{i2}+\lambda_{i1})$
\\
\\
接着,先求沿约束方向未考虑不等式$0 \leq \lambda_i \leq C,i=1,2$约束
时$\lambda_{i2}$的最优解$\lambda_{i2}^{*,unc}$
\\
再求考虑不等式$0 \leq \lambda_i \leq C,i=1,2$约束后的
解$\lambda_{i2}^*$

令$\displaystyle{g(x)=\sum_{i= 1}\lambda_iy_i(\textbf{\emph{x}}_i·\textbf{\emph{x}})+b}$

$\displaystyle{E_i=g(x_i)-y_i=\left(
        \sum_{i= 1}\lambda_iy_i(\textbf{\emph{x}}_i·\textbf{\emph{x}})+b
        \right)-y_i,i=1,2}$

$\displaystyle{v_i=\sum_{j\neq 1,2}y_i\lambda_i
        (\textbf{\emph{x}}_i·\textbf{\emph{x}}_j)
        =g(x_i)
        -\sum_{j= 1}^2 \lambda_iy_i(\textbf{\emph{x}}_i·\textbf{\emph{x}}_j)-b,i=1,2}$
\\
\\
目标函数可写成

\begin{center}

    $\displaystyle{
            W(\lambda_{i1},\lambda_{i2})=
            \frac{1}{2}(\textbf{\emph{x}}_1·\textbf{\emph{x}}_1)
            \lambda_{i1}^2
            +
            \frac{1}{2}(\textbf{\emph{x}}_2·\textbf{\emph{x}}_2)
            \lambda_{i2}^2
            +
            y_1 y_2 (\textbf{\emph{x}}_1·\textbf{\emph{x}}_2)
            \lambda_{i1}\lambda_{i2}}$
    \\
    $\displaystyle{-
            (\lambda_{i1}+\lambda_{i2})
            +
            y_1 \lambda_{i1} v_1
            +
            y_2 \lambda_{i2} v_2
        }$

\end{center}

由$\lambda_{i1} y_1=\varsigma -\lambda_{i2} y_2,y_i^2=1$,
故$\lambda _{i1}=(\varsigma-\lambda_{i2} y_2)y_1$
\\
代入$ W(\lambda_{i1},\lambda_{i2})$得

\begin{center}

    $\displaystyle{
            W(\lambda_{i2})=
            \frac{1}{2}(\textbf{\emph{x}}_1·\textbf{\emph{x}}_1)
            (\varsigma-\lambda_{i2} y_2)^2
            +
            \frac{1}{2}(\textbf{\emph{x}}_2·\textbf{\emph{x}}_2)
            \lambda_{i2}^2
            +
            y_2 (\textbf{\emph{x}}_1·\textbf{\emph{x}}_2)
            (\varsigma-\lambda_{i2} y_2)\lambda_{i2}}$
    \\
    $\displaystyle{-
            (\varsigma-\lambda_{i2} y_2)y_1
            -
            \lambda_{i2}
            +
            (\varsigma-\lambda_{i2} y_2) v_1
            +
            y_2 \lambda_{i2} v_2
        }$

\end{center}

对$\lambda_{i2}$求导并令其等于零

\begin{center}

    $\displaystyle{
            \frac{\partial W}{\partial \lambda_{i2}}=
            (\textbf{\emph{x}}_1·\textbf{\emph{x}}_1)
            \lambda_{i2}
            +
            (\textbf{\emph{x}}_2·\textbf{\emph{x}}_2)
            \lambda_{i2}
            -
            2(\textbf{\emph{x}}_1·\textbf{\emph{x}}_2)
            \lambda_{i2}}
        -
        (\textbf{\emph{x}}_1·\textbf{\emph{x}}_1)
        \varsigma \lambda_{i2}
    $
    \\

    $\displaystyle{
            +
            (\textbf{\emph{x}}_1·\textbf{\emph{x}}_2)
            \varsigma \lambda_{i2}
            +y_1y_2-1
            -y_2  v_1
            +y_2  v_2
            =0
        }$

\end{center}
$\displaystyle{
        ((\textbf{\emph{x}}_1·\textbf{\emph{x}}_1)
        +(\textbf{\emph{x}}_2·\textbf{\emph{x}}_2)
        -2(\textbf{\emph{x}}_1·\textbf{\emph{x}}_2)
        )\lambda_{i2}}$
\\
$\displaystyle{
        =y_2(y_2-y_1+\varsigma
        (\textbf{\emph{x}}_1·\textbf{\emph{x}}_1)
        -\varsigma
        (\textbf{\emph{x}}_1·\textbf{\emph{x}}_2)
        +v_1-v_2)}$
\\\quad
\\\quad
$\displaystyle{
    =y_2 \Biggl[
    y_2-y_1+\varsigma
    (\textbf{\emph{x}}_1·\textbf{\emph{x}}_1)
    -\varsigma
    (\textbf{\emph{x}}_1·\textbf{\emph{x}}_2)
    +\left(g(x_1)
    -\sum_{j= 1}^2 \lambda_jy_j(\textbf{\emph{x}}_1·\textbf{\emph{x}}_j)-b\right)
    }$
\\
$\displaystyle{
        -\left(g(x_2)
        -\sum_{j= 1}^2 \lambda_jy_j(\textbf{\emph{x}}_2·\textbf{\emph{x}}_j)-b\right)
        \Biggr]}$
\\
\\
代入$\varsigma =\lambda_{i1}y_1+\lambda_{i2}y_2$得
$\displaystyle{
        ((\textbf{\emph{x}}_1·\textbf{\emph{x}}_1)
        +(\textbf{\emph{x}}_2·\textbf{\emph{x}}_2)
        -2(\textbf{\emph{x}}_1·\textbf{\emph{x}}_2)
        )\lambda_{i2}^{*,unc}}$

$\displaystyle{
        =y_2(((\textbf{\emph{x}}_1·\textbf{\emph{x}}_1)
        +(\textbf{\emph{x}}_2·\textbf{\emph{x}}_2)
        -2(\textbf{\emph{x}}_1·\textbf{\emph{x}}_2)
        )\lambda_{i2}y_2+y_2-y_1+g(x_1)+g(x_2))
    }$

$\displaystyle{
        =((\textbf{\emph{x}}_1·\textbf{\emph{x}}_1)
        +(\textbf{\emph{x}}_2·\textbf{\emph{x}}_2)
        -2(\textbf{\emph{x}}_1·\textbf{\emph{x}}_2)
        )\lambda_{i2}+y_2(E_1-E_2)
    }$
\\
\\
令$\eta =((\textbf{\emph{x}}_1·\textbf{\emph{x}}_1)
    +(\textbf{\emph{x}}_2·\textbf{\emph{x}}_2)
    -2(\textbf{\emph{x}}_1·\textbf{\emph{x}}_2)
    )$
得到
\begin{center}
    $\displaystyle{\lambda_{i2}^{*,unc}=\lambda_{i2}
            +\frac{y_2(E_1-E_2)}{\eta}}$
\end{center}

为满足不等式约束条件,将上式限制在$[L,H]$间,可得$\lambda_{i2}^*$表达式

\begin{center}
    $$ \lambda_{i2}^*=\left\{
        \begin{array}{rcl}
             & H                    ,\quad & {\lambda_{i2}^{*,unc}>H}             \\
             & \lambda_{i2}^{*,unc} ,\quad & {L \leq \lambda_{i2}^{*,unc} \leq H} \\
             & L                    ,\quad & {\lambda_{i2}^{*,unc}<L}
        \end{array}
        \right. $$

    $\lambda_{i1}^*=\lambda_{i1}+y_1y_2(\lambda_{i2}-\lambda_{i2}^*)$
\end{center}
%\begin(center)
%$\displaystyle{=y_2(
%        ((\textbf{\emph{x}}_1·\textbf{\emph{x}}_1)
%        +(\textbf{\emph{x}}_2·\textbf{\emph{x}}_2)
%        -2(\textbf{\emph{x}}_1·\textbf{\emph{x}}_2)
%       )\lambda_{i2}y_2+y_2-y_1+g(x_1)+g(x_2))}$
%\end(center)

%SMO不断执行下述步骤:

%(1)选一对需要更新的变量$\lambda _i,\lambda _j$

%(2)固定$\lambda _i,\lambda _j$以外的参数,
%求解$\displaystyle{\mathop{max}\limits_{\lambda_{i}\geq 0}\left\{\sum_{i}\lambda _{i}-
%    \frac{1}{2}
%    (\sum_{i}\sum_{j}\lambda_{i}\lambda_{j}
%    y_{i}y_{j}(\textbf{$x_{i}$}·\textbf{$x_{j}$}))\right\}}$获得更新后的$\lambda _i,\lambda_j$\\
%\\
%只要选$\lambda_i,\lambda_j$中一个不满足KKT条件,即
%$$ \left\{
%    \begin{array}{lr}

%        \lambda _i \geq 0                                                          \\
%        y_i f(x_i) \\
%        $$\sum\limits_{i=1}\limits^6$$(y_{i}-w_{2}x_{i}^2-w_{1}x_{i}-w_{0})=0
%    \end{array}
%    \right. $$

\end{document}