\documentclass[UTF8]{ctexart}
\usepackage{graphicx}
\usepackage{geometry}
\geometry{a4paper,scale=0.85}
\usepackage{amsmath}
\usepackage{xcolor}  	%高亮使用的颜色
\definecolor{commentcolor}{RGB}{85,139,78}
\definecolor{stringcolor}{RGB}{206,145,108}
\definecolor{keywordcolor}{RGB}{34,34,250}
\definecolor{backcolor}{RGB}{220,220,220}
\usepackage{accsupp}
\newcommand{\emptyaccsupp}[1]{\BeginAccSupp{ActualText={}}#1\EndAccSupp{}}
\usepackage{listings}
%\documentclass{article}
\CTEXsetup[format={\large\bfseries}]{section}
%由小到大分别是: 
%\tiny \scriptsize \footnotesize \small \normalsize 
%\large \Large \LARGE \huge \Huge
\usepackage{xeCJK} % CJK语言环境,使用XeLaTex进行编译
\usepackage{authblk} % 对应中文部分的作者机构特殊语法
\author{asandstar@github}
\title{HW1}
\begin{document}
\maketitle
\section{简述机器学习的类别,监督学习过程中涉及哪些要素}
 (1)机器学习可以分为监督学习、无监督学习、强化学习、半监督学习与主动学习。

监督学习指从标注数据中学习预测模型的机器学习问题。

无监督学习指从无标注数据中学习预测模型的机器学习问题。

强化学习指智能系统在与环境的连续互动中学习最优行为策略的机器学习问题。

半监督学习指利用标注数据和未标注数据学习预测模型的机器学习问题。

主动学习指机器不断给出实例让教师进行标注,再利用标注数据学习预测模型的机器学习问题。

(2)监督学习过程中涉及的要素为数据、模型、策略和算法。其中三要素为:模型、策略和算法。
\pagestyle{plain}
\section{现有以下数据点($x_{i}$,$y_{i}$)
  ,试用二次函数拟合,在最大似然准则下估
  计系数,并计算拟合残差\\
  (0,-0.13),
  (0.2,0.09),
  (0.4,0.59),
  (0.6,1.62),
  (0.8,2.5),
  (1,3.4)}

已知$X=[0\ 0.2\ 0.4\ 0.6\ 0.8\ 1]$,

$t=[-0.13\ 0.09\ 0.59\ 1.62\ 2.5\ 3.4]$

%$E(w)=\frac{1}{2} $$\sum\limits_{i=1}\limits^6$$ (y(x_{i},w)-t_{i})^2$
% =\frac{1}{2} \mathbf{(\textbf{\emph{y-t}})}^\top (\textbf{\emph{y-t}})
% =\frac{1}{2} \mathbf{(\textbf{\emph{Xw-t}})}^\top (\textbf{\emph{Xw-t}})$

$w*=arg \mathop{min}\limits_{w}$$\sum\limits_{i=1}\limits^6$$ (f(x_{i},w)-y_{i})^2$

%令$\displaystyle{\frac{ \partial \emph {E} }{ \partial \textbf{\emph {w} }}=0}$

%则$\displaystyle{\frac{ \partial \emph {E} }{ \partial \textbf{\emph {w}}}
%    \\=\frac{1}{2} \frac{ \partial \mathbf{(\textbf{\emph{Xw-t}})}^\top (\textbf{\emph{Xw-t}}) }
%    {\partial \textbf{\emph {w}} }
%    \\=\textbf{\emph{X}}^\top(\textbf{\emph{Xw-t}})
%    \\=0}$

%所以$\textbf{\emph{X}}^\top \textbf{\emph{Xw}}=\textbf{\emph{X}}^\top \textbf{\emph{t}}$

%$\textbf{\emph{w}}=\mathbf{(\textbf{\emph{X}}^\top \textbf{\emph{X}})}^{-1}\textbf{\emph{X}}^\top\textbf{\emph{t}}$


$ $$\sum\limits_{i=1}\limits^6$$
  \displaystyle{\frac{ \partial (f(x_{i},w)-y_{i})^2 }{ \partial \emph {w} }}
  =2$$\sum\limits_{i=1}\limits^6$$(f(x_{i},w)-y_{i})
  \displaystyle{\frac{ \partial f(x_{i},w) }{ \partial \emph {w} }=0}$

由于$\displaystyle{\frac{ \partial f(x_{i},w) }{ \partial \emph {w} }=[x_{i}^2,x_{i},1]}$
$$ \left\{
  \begin{array}{lr}

    $$\sum\limits_{i=1}\limits^6$$(y_{i}-w_{2}x_{i}^2-w_{1}x_{i}-w_{0})x_{i}^2=0 \\
    $$\sum\limits_{i=1}\limits^6$$(y_{i}-w_{2}x_{i}^2-w_{1}x_{i}-w_{0})x_{i}=0   \\
    $$\sum\limits_{i=1}\limits^6$$(y_{i}-w_{2}x_{i}^2-w_{1}x_{i}-w_{0})=0
  \end{array}
  \right. $$

$$
  \begin{bmatrix}
    \sum\limits_{i=1}\limits^6 y_{i}x_{i}^2 \\
    \sum\limits_{i=1}\limits^6 y_{i}x_{i}   \\
    \sum\limits_{i=1}\limits^6 y_{i}
  \end{bmatrix}
  =
  \begin{bmatrix}
    \sum\limits_{i=1}\limits^6 x_{i}^4 & \sum\limits_{i=1}\limits^6 x_{i}^3 & \sum\limits_{i=1}\limits^6 x_{i}^2 \\
    \sum\limits_{i=1}\limits^6 x_{i}^3 & \sum\limits_{i=1}\limits^6 x_{i}^2 & \sum\limits_{i=1}\limits^6 x_{i}   \\
    \sum\limits_{i=1}\limits^6 x_{i}^2 & \sum\limits_{i=1}\limits^6 x_{i}   & 6
  \end{bmatrix}
  ·
  \begin{bmatrix}
    \emph{$w_{2}$} \\
    \emph{$w_{1}$} \\
    \emph{$w_{0}$}
  \end{bmatrix}
$$
\\
\\
\lstset{						%高亮代码设置
  language=python, 					%Python语法高亮
  linewidth=0.9\linewidth,      		%列表list宽度
  %basicstyle=\ttfamily,				%tt无法显示空格
  commentstyle=\color{commentcolor},	%注释颜色
  keywordstyle=\color{keywordcolor},	%关键词颜色
  stringstyle=\color{stringcolor},	%字符串颜色
  %showspaces=true,					%显示空格
  numbers=left,						%行数显示在左侧
  numberstyle=\tiny\emptyaccsupp,		%行数数字格式
  numbersep=5pt,						%数字间隔
  frame=single,						%加框
  framerule=0pt,						%不划线
  escapeinside=@@,					%逃逸标志
  emptylines=1,						%
  xleftmargin=3em,					%list左边距
  backgroundcolor=\color{backcolor},	%列表背景色
  tabsize=4,							%制表符长度为4个字符
  gobble=4							%忽略每行代码前4个字符
}

\begin{lstlisting}
    import numpy as np
    from numpy import *
    res1 = res2 = 0
    m1 = m2 = m3 = 0
    count1 = count2 = count3 =0
    count4 = count5 = 0
    x = [0, 0.2, 0.4, 0.6, 0.8, 1]
    y = [-0.13, 0.09, 0.59, 1.62, 2.5, 3.4]
    for i in range(6):
        res1 = y[i] * x[i] * x[i]
        res2 = y[i] * x[i]
        m1 = x[i] * x[i] * x[i] * x[i]
        m2 = x[i] * x[i] * x[i]
        m3 = x[i] * x[i]
        count1 = count1 + res1
        count2 = count2 + m1
        count3 = count3 + m2
        count4 = count4 + res2
        count5 = count5 + m3
    res3 = sum(y)
    m4 = sum(x)
    m5 = 6
    yx = np.array([[count1], [count4], [res3]])
    xx = np.array([[count2, count3, count5], 
                 [count3, count5, m4], 
                 [count5, m4, m5]])
    xx_inv = np.linalg.inv(xx)
    w2 = w1 = w0 = 0
    w = [w2, w1, w0]
    w = np.dot(xx_inv, yx)
    print(w)
    y_see = np.array(y)
    s = 0
    y_p = {}
    for i in range(6):
        y_p[i] = w[0]*x[i]*x[i]+w[1]*x[i]+w[2]
        s = s+y_see[i]-y_p[i]
    print(s)
\end{lstlisting}

解得$w_{2}=2.19642857,\ w_{1}=1.505,\ w_{0}=-0.21285714$

$y(x,\textbf{\emph{w}})=-0.21285714+1.505x+2.19642857x^2$

拟合残差$\varepsilon =\sum\limits_{i=1}\limits^6 (t-y(x_{i},\textbf{\emph{w}}))$

计算得$\varepsilon=3.06421555×10^{-14}$

\section{题2中设$y_{i}$的观测误差服从高斯分布N(0,0.1),二次函数的各系数均
  服从先验分布N(0,0.3),试在最大后验准则下重新估计各系数}
$y_{i}=f(w,x)+\delta$

$\delta  \sim N(0,\sigma _{1}),\ \sigma _{1}=0.1$

$w \sim N(0,\sigma _{2}),\ \sigma _{2}=0.3$

后验概率$\displaystyle{P(w)=exp(-\frac{w^2}{2\sigma _{2}^2})\prod_{i=1}^6 exp(-\frac{(y_{i}-f(w,x_{i}))^2}{2\sigma _{1}^2})}$

两边取对数后对$\emph{$w_{j}$}$求导得
$\displaystyle{\frac{\text{d}ln\ P}{\text{d}\emph{$w_{j}$}}
  =\frac{\text{d}[(-\frac{\pmb{\emph{w}}^2}{2\sigma _{2}^2})+
  \sum_{i=1}^6 (-\frac{(y_{i}-f(\pmb{\emph{w}},x_{i}))^2}{2\sigma _{1}^2})]}
  {\text{d}\emph{$w_{j}$}}}$

$\displaystyle{
  =\frac{\text{d}[(-\frac{\pmb{\emph{w}}^2}{2\sigma _{2}^2})+
  \sum_{i=1}^6 (-\frac{(y_{i}-w_{2}x_{i}^2-w_{1}x_{i}-w_{0})^2}{2\sigma _{1}^2})]}
  {\text{d}\emph{$w_{j}$}}
  }$

令上式$=0$解得

$\sum_{i=1}^6 x_{i}^2(y_{i}-w_{2}x_{i}^2-w_{1}x_{i}-w_{0})-\frac{\sigma_{1}^2}
  {\sigma_{2}^2}w_{2}=0$

$\sum_{i=1}^6 x_{i}(y_{i}-w_{2}x_{i}^2-w_{1}x_{i}-w_{0})-\frac{\sigma_{1}^2}
  {\sigma_{2}^2}w_{1}=0$,

$\sum_{i=1}^6(y_{i}-w_{2}x_{i}^2-w_{1}x_{i}-w_{0})-\frac{\sigma_{1}^2}
  {6\sigma_{2}^2}w_{0}=0$

$$
  \begin{bmatrix}
    \sum\limits_{i=1}\limits^6 y_{i}x_{i}^2 \\
    \sum\limits_{i=1}\limits^6 y_{i}x_{i}   \\
    \sum\limits_{i=1}\limits^6 y_{i}
  \end{bmatrix}
  =
  \begin{bmatrix}
    \sum\limits_{i=1}\limits^6 x_{i}^4 +\frac{\sigma_{1}^2}{\sigma_{2}^2} & \sum\limits_{i=1}\limits^6 x_{i}^3                                   & \sum\limits_{i=1}\limits^6 x_{i}^2   \\
    \sum\limits_{i=1}\limits^6 x_{i}^3                                    & \sum\limits_{i=1}\limits^6 x_{i}^2+\frac{\sigma_{1}^2}{\sigma_{2}^2} & \sum\limits_{i=1}\limits^6 x_{i}     \\
    \sum\limits_{i=1}\limits^6 x_{i}^2                                    & \sum\limits_{i=1}\limits^6 x_{i}                                     & 6+\frac{\sigma_{1}^2}{6\sigma_{2}^2}
  \end{bmatrix}
  ·
  \begin{bmatrix}
    \emph{$w_{2}$} \\
    \emph{$w_{1}$} \\
    \emph{$w_{0}$}
  \end{bmatrix}
$$

上述代码需要修改的部分为24至26行
\begin{lstlisting}
    xx = np.array([[count2+1/9, count3, count5], 
                   [count3, count5+1/9, m4], 
                   [count5, m4, m5+1/(6*9)]])
\end{lstlisting}

带入数值得$w_{2}=1.82312499,\ w_{1}=1.6194856,\ w_{0}=-0.13281205$

$y(x,\textbf{\emph{w}})=-0.13281205+1.6194856x+1.82312499x^2$

拟合残差$\varepsilon =\sum\limits_{i=1}\limits^6 (t-y(x_{i},\textbf{\emph{w}}))$

计算得$\varepsilon=-0.00245948$

\section{试画出下式的计算图,并在该图上用链式法则计算梯度
$\frac{ \partial f }{ \partial x }$ :
$f(x)=e^{ax}+e^{bx^2+cx}+\sqrt{x^2+cosx}$}
\includegraphics[width = .9\textwidth]{graph.png}
\begin{table}[!h]
  \renewcommand{\arraystretch}{1.2}
  \label{table_example}
  \centering
  \begin{tabular}{c|c c c c}
    \hline
    $f(x) $                            & $e^{ax} $ & $e^{bx^2+cx}$        & $\sqrt{x^2+cosx}$                  \\
    \hline
    $\frac{\partial f(x)}{\partial x}$ & $ae^{ax}$ & $(2bx+c)e^{bx^2+cx}$ & $\frac{2x-sinx}{2\sqrt{x^2+cosx}}$ \\
    \hline
  \end{tabular}
\end{table}

$\frac{ \partial f }{ \partial x }=ae^{ax}+(2bx+c)e^{bx^2+cx}+\frac{2x-sinx}{2\sqrt{x^2+cosx}}$


\section{试证明以下矩阵代数的等式
  \\
  \\
  $\displaystyle{\frac{ \partial \mathbf{\emph {a}}^\top  \emph {x}  }{ \partial \emph {x} }=\mathbf{\emph {a}}^\top }$
  \\
  \\
  $\displaystyle{\frac{ \partial \mathbf{\emph {a}}^\top  \emph {Xb}  }{ \partial \emph {X} }=\emph {a}\mathbf{\emph {b}}^\top }$
  \\
  \\
  $\displaystyle{\frac{ \partial \mathbf{\emph{x}}^\top  \emph {Bx}  }{ \partial \emph {x} }=\mathbf{\emph{x}}^\top (\emph {B}+\mathbf{\emph{B}}^\top)}$
  \\
 }
证明:
\\
\\
(1)

$\mathbf{\textbf {\emph{a}}}^\top=[\emph {$a_{1}$\ $a_{2}$\ $a_{3}$\ …\ $a_{n}$}],
  \ x=\mathbf{[\emph {$x_{1}$\ $x_{2}$\ $x_{3}$\ …\ $x_{n}$}]}^\top$

因为$\displaystyle{
    \frac{ \partial \mathbf{\textbf {\emph{a}}}^\top  \textbf {\emph{x}}  }
    { \partial \emph {$x_{i}$} } =
    \frac{ \partial ( \sum\limits_{j=1}\limits^n  \emph {$a_{j}$}  \emph {$x_{j}$}  )}
    { \partial \emph {$x_{i}$} }=
    \textbf{\emph {$a_{i}$}}}$

%其中
%$\mathbf{\textbf{\emph {a}}}^\top $
%在第i列


所以$\displaystyle{\frac{ \partial \mathbf{\textbf {\emph {a}}}^\top  \textbf {\emph {x}}  }
{ \partial \textbf{\emph {x}} }=
      %\sum\limits_{k=0}\limits^n \mathbf{\pmb{\emph {$a_{k}$}}}^\top =
      [\emph {$a_{1}$\ $a_{2}$\ $a_{3}$\ …\ $a_{n}$}]=
    \mathbf{\textbf{\emph {a}}}^\top}$

证毕
\\
\\
(2)

$\mathbf{\textbf {\emph{a}}}^\top=[\emph {$a_{1}$\ $a_{2}$\ $a_{3}$\ …\ $a_{m}$}],
  \ \textbf {\emph{b}}=\mathbf{[\emph {$b_{1}$\ $b_{2}$\ $b_{3}$\ …\ $b_{n}$}]}^\top$

因为$\displaystyle{
    \frac{ \partial \mathbf{\textbf {\emph{a}}}^\top  \textbf {\emph{X}} \textbf {\emph{b}} }
    { \partial \emph {$X_{ij}$} } =
    \frac{ \partial (\sum\limits_{t=1}\limits^m \sum\limits_{k=1}\limits^n \emph {$a_{t}$}  \emph {$X_{tk}$} \emph {$b_{k}$} )}
    { \partial \emph {$X_{ij}$} }= \emph {$a_{i}$} \emph {$b_{j}$} }$

所以
$$
  \displaystyle{\frac{ \partial \mathbf{\textbf {\emph {a}}}^\top  \textbf {\emph{X}} \textbf {\emph{b}} }
    { \partial \textbf{ \emph {X} }}=}
  \begin{bmatrix}
    a_{1}  \\
    a_{2}  \\
    \vdots \\
    a_{m}
  \end{bmatrix}
  \begin{bmatrix}
    b_{1} & b_{2} & \dots & b_{n}
  \end{bmatrix}
  =\textbf{\emph {a}}\mathbf{\textbf{\emph {b}}}^\top
$$

证毕
\\
\\
(3)

$B_{n×n}=(b_{ij})_{i=1,j=i}^{n,n},\ x=\mathbf{[\emph {$x_{1}$\ $x_{2}$\ $x_{3}$\ …\ $x_{n}$}]}^\top$

因为
$ \displaystyle{
    \frac{ \partial \mathbf{\textbf {\emph{x}}}^\top  \textbf {\emph{B}} \textbf {\emph{x}} }
    { \partial \emph {$x_{ij}$} } =
    \frac{ \partial (\sum\limits_{t=1}\limits^n \sum\limits_{k=1}\limits^n \emph {$b_{tk}$}\emph {$x_{t}$} \emph {$x_{k}$} )}
    { \partial \emph {$x_{t}$} } =
    \sum\limits_{k=1}\limits^n \emph {$b_{tk}$} \emph {$x_{k}$}
  }$

所以
$$
  \displaystyle{\frac{ \partial \mathbf{\textbf{\emph{x}}}^\top \textbf{ \emph {Bx} } }{ \partial \textbf{\emph {x} }}
    =}
  \begin{bmatrix}
    b_{11}x_{1}+ b_{12}x_{2}+ & \dots & +b_{1n}x_{n} \\
    b_{21}x_{1}+ b_{22}x_{2}+ & \dots & +b_{2n}x_{n} \\
    \vdots                                           \\
    b_{n1}x_{1}+ b_{n2}x_{2}+ & \dots & +b_{nn}x_{n}
  \end{bmatrix}
  +
  \begin{bmatrix}
    b_{11}x_{1}+ b_{21}x_{2}+ & \dots & +b_{n1}x_{n} \\
    b_{12}x_{1}+ b_{22}x_{2}+ & \dots & +b_{n2}x_{n} \\
    \vdots                                           \\
    b_{1n}x_{1}+ b_{2n}x_{2}+ & \dots & +b_{nn}x_{n}
  \end{bmatrix}
$$
\\
$$
  =
  \mathbf{\begin{bmatrix}
      x_{1}  \\
      x_{2}  \\
      \vdots \\
      x_{n}
    \end{bmatrix}}^\top
  \begin{bmatrix}
    b_{11} & b_{12} & \dots  & b_{1n} \\
    b_{21} & b_{22} & \dots  & b_{2n} \\
    \vdots & \vdots & \ddots & \vdots \\
    b_{n1} & b_{n2} & \dots  & b_{nn}
  \end{bmatrix}
  +
  \mathbf{\begin{bmatrix}
      x_{1}  \\
      x_{2}  \\
      \vdots \\
      x_{n}
    \end{bmatrix}}^\top
  \begin{bmatrix}
    b_{11} & b_{21} & \dots  & b_{n1} \\
    b_{12} & b_{22} & \dots  & b_{n2} \\
    \vdots & \vdots & \ddots & \vdots \\
    b_{1n} & b_{2n} & \dots  & b_{nn}
  \end{bmatrix}
  =\textbf{$\mathbf{\emph{x}}^\top (\emph {B}+\mathbf{\emph{B}}^\top)$}$$

证毕
\end{document}